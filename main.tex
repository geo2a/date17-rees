\documentclass[conference]{IEEEtran}
\usepackage[pdftex]{graphicx}
\graphicspath{{../pdf/}{../jpeg/}}
\DeclareGraphicsExtensions{.pdf,.jpeg,.png}
\usepackage{url}

% correct bad hyphenation here
\hyphenation{op-tical net-works semi-conduc-tor}

\begin{document}
\title{Prototyping Resilient Processing Cores in Workcraft}

\vspace{-1mm}
\author{\IEEEauthorblockN{Georgy Lukyanov\IEEEauthorrefmark{2},
Alessandro de Gennaro\IEEEauthorrefmark{3},
Andrey Mokhov\IEEEauthorrefmark{3},
Paulius Stankaitis\IEEEauthorrefmark{3},
Maxim Rykunov\IEEEauthorrefmark{4}}
\IEEEauthorblockA{\IEEEauthorrefmark{2}Southern Federal University, Rostov-on-Don, Russia}
\IEEEauthorblockA{\IEEEauthorrefmark{3}Newcastle University, Newcastle upon Tyne, UK}
\IEEEauthorblockA{\IEEEauthorrefmark{4}IMEC, Leuven, Belgium}
\vspace{-6mm}
}

\maketitle

\begin{abstract}
We present a methodology for the design and fast prototyping of
processing cores with resilient microarchitecture. The resilience is
achieved by equipping the core with a family of datapath components
optimised for different operating modes and a flexible control
structure that allows to change an instruction implementation at
runtime depending on current conditions and application
requirements. We use asynchronous design techniques to achieve
\emph{short-term resilience}, i.e. survival in extreme environmental
conditions, such as near-threshold or unstable voltage supply.
\emph{Long-term resilience} is achieved through
runtime reconfiguration of the processor microarchitecture, which
is essential for safety-critical applications that cannot be taken
offline for maintenance, such as biomedical implants. By using
formal methods one can guarantee the correctness and uninterrupted
service during such runtime reconfigurations.

The presented methodology is supported by open-source tool
\textsc{Workcraft}, and has been validated by fabricating
two ASICs: an Intel 8051 processing core and a
reconfigurable dataflow accelerator.
To facilitate fast prototyping of resilient processing cores, we
introduce a domain-specific language for their formal specification,
software-level simulation and hardware synthesis.
\end{abstract}

% no keywords

\IEEEpeerreviewmaketitle
\section{Formal methods for resilient systems}
\vspace{-1mm}

Many resilient systems rely on runtime reconfigurability to adapt to
continuously changing environment without any human intervention. For example,
biomedical implants must be able to operate autonomously within patients,
adapting to short-term and long-term changes, with required lifetimes in the
order of decades. Runtime reconfigurability can be achieved both in hardware
and software; the latter is less challenging to implement, however, the former
is often unavoidable. In this paper we focus on hardware reconfigurability.

Formal methods provide a systematic approach for developing complex systems
in a reusable and correct-by-construction manner. The suitability of these
techniques for the specification and verification of reconfigurable
systems has been studied for some time now. We refer the reader to a survey by
Calinescu and Kikuchi~\cite{calinescu2010formal} that overviews the use of
formal methods for adaptive systems at runtime, and an empirical comparison
of different modelling formalisms by
Bhattacharyya et al.~\cite{DBLP:journals/corr/BhattacharyyaMP16}. In our
work~\cite{ISA-formal} we introduced a methodology for the formal specification
and synthesis of processor microarchitecture with runtime reconfigurability.
The methodology is supported by open-source EDA toolsuite
\textsc{Workcraft}~\cite{workcraft_web} and has been validated in silicon~\cite{rec-proc}.
In the rest of the paper we discuss our current work on
providing fast prototyping of reconfigurable resilient processing cores in
\textsc{Workcraft} using a domain-specific language for
microarchitecture description (Section~\ref{sec:dsl}), and report on our
experience in designing resilient processing cores using this methodology
(Section~\ref{sec:experience}).

% The authors of the paper~\cite{Weyns:2012:SFM:2347583.2347592}
% review a decade of research on formal
% methods for adaptive systems. Even though the formal verification is traditionally
% an off-line procedure some progress have been achieved in integrating techniques like
% model checking at runtime. The model checking approach involves specifying system behaviour
% formally or else using a model finding approach \cite{kikuchi2010configuration}
% and exhaustively checking that systems properties hold at every system change.
% The counter-examples can be used to resolve the policy conflicts as discussed by
% Calinescu and Kikuchi in their paper. The other methods include using a mathematical logic and theorem prover
% to verify dynamic reconfiguration of cores at a runtime \cite{singh1999formal}. The proposed schema
% automatically generates the new specification of the system in a propositional logic and calls
% the theorem prover at the runtime to check the correctness.

% A lot of safety critical systems
% system has to preserve requirements for resources in case of system failures and in many instances this requires
% reasoning in discrete-continuous domain which is best captured with hybrid models.
% A generic substitution model has been introduced by Babin et al. to reason about reconfigurable
% systems where system state has to be preserved \cite{Babin2016}. The biggest challenge in
% integrating formal verification at runtime is the tool scalability as particularly for
% model checking state-space explosion remains a big challenge.

 %suitable models for analysing
 %discrete-continuous systems are hybrid models. A lot of work has been done to allow
 %modelling such a complex systems and one of the generic models for reconfigurable
 %systems is presented by Babin et al. with a case study for maintaining safe energy
 %level \cite{Babin2016}. The future work direction could involve integrating a similiar
 %approach within our graph-based model.
%Runtime systems adaptation is currently one the biggest challenges in formal
%methods community.
%A number of formal models exists for specification of process-based systems for
%instance Petri nets \cite{peterson1981petri} or Finite State Machines
%\cite{nowick1993automatic} or else hardware description languages such as Verilog or
% VHDL. The majority of industrial projects used formal techniques exactly for the
%specification and modelling as has been shown by Woodcock et al.
%\cite{woodcock2009formal}. The description of application (even formal) is not
%ufficient to evidently demonstrate reliability of the system. The development
%process must be accompanied with a form formal verification and mainly there exists
%two formal verification techniques - theorem proving and model checking. The
%principal of the model checking is to construct a finite-state model and by
%exploring state-space exhaustively check whether the model meets requirements. The
%alternative technique for automated verification is automated theorem proving which
%does not rely on exploring the state-space. The benefits of automated theorem
%proving in hardware design validation are increasingly recognised by the embedded
%system industry \cite{clarke1996formal} and several companies have already
%integrated tools like ACL2 or Isabelle into application development process
%\cite{kaufmann2004some}.

%Our graph-based
%approach combines advantages of PNs and FSMs are is able of capturing concurrency
%and choice in a compact and efficient way.

\section{DSLs for improved design productivity\label{sec:dsl}}

\emph{Domain-Specific Languages} (DSLs) are designed to have a maximal expression for
tasks in a particular domain (for example, VHDL for hardware description or
\LaTeX~for typesetting). However, implementing a language from scratch may be tedious,
time-consuming and error-prone. Therefore, DSLs are often embedded into existing general-purpose programming languages, which is particularly convenient for
prototyping purposes.
Modern functional programming languages such as Haskell offer a
wide range of facilities for construction of \emph{Embedded Domain-Specific Languages}
(EDSLs) that benefit from features of lightweight formal verification provided by
the rich type system and highly-tailored syntax achieved using various functional
programming idioms~\cite{HudakDSLs}.

\begin{figure*}[ht!]
\begin{center}
    \includegraphics[width=0.96\linewidth]{FIG/screen.png}
    \vspace{-3.5mm}
    \caption{From formal specification to hardware synthesis and
    simulation of a simple 3-instruction processing core. A screenshot of \textsc{Workcraft}.}
    \label{fig:screenshot}
\end{center}
\vspace{-7.5mm}
\end{figure*}

To design resilient and reconfigurable systems, it is vital to have formal
specification methods, simulation facilities and verification techniques.
EDSLs can increase the productivity at every stage of hardware design:
high-level specification
languages help to describe the system functionality in a declarative way, software
simulation environments allow to evaluate the system capabilities
without fabricating an expensive prototype, and advanced types of the host language
provide compiler-checked correctness guarantees for synthesis.

The \textsc{Workcraft} framework provides three DSLs:

\begin{itemize}
\item Signal Transition Graphs (STGs), a signal-level DSL for
specifying resilient asynchronous controllers~\cite{STG}.
\item Conditional Partial Order Graphs (CPOGs), a DSL for specifying
reconfigurable processor microarchitectures supported by optimal
instruction encoding algorithms~\cite{ISA-formal}.
\item Dataflow Structures (DFSs), a dataflow-level DSL
for specifying dataflow computation graphs~\cite{DFS}.
\end{itemize}

\textsc{Workcraft} can synthesise and export models described in these DSLs into
Verilog, a low-level language for hardware description, supported
by conventional EDA tool chain.

In our work, we focus on bridging Event-B~\cite{EventB}, a high-level formal notation
for the specification of system requirements and reconfiguration,
and DSLs provided by \textsc{Workcraft}. As a prototype of a bridging language,
we present \emph{Farfalle}, an intermediate-level DSL embedded in Haskell for
the description of reconfigurable processor microarchitectures.
Haskell provides powerful abstractions, e.g. the \texttt{Monad} type
class~\cite{WadlerMonads},
that help to build custom EDSLs with strong static typing and compositional
semantics.

Here are two simple examples of \emph{Farfalle} code, highlighting the power of
Haskell's monadic \texttt{do}-notation for building composable EDSL programs in clear
imperative step-by-step manner. To \emph{increment a register}, one needs to
read the value it contains and then write the incremented value back:

\vspace{-2mm}
\begin{verbatim}
increment register = do
    value <- readRegister register
    writeRegister register (value + 1)
\end{verbatim}
\vspace{-1mm}

\noindent
To \emph{fetch an immediate argument}, one needs to
increment the program counter \texttt{pc} and fetch the value it points to. Note
how easily one can reuse previously defined \texttt{increment} function:

\vspace{-2mm}
\begin{verbatim}
fetchArgument = do
    increment pc
    address <- readRegister pc
    readMemory address
\end{verbatim}
\vspace{-1mm}

\emph{Farfalle} code can be translated both to Event-B notation for formal
verification and to one of \textsc{Workcraft}'s DSLs, e.g. a CPOG model for
further hardware synthesis, as illustrated in Fig.~\ref{fig:screenshot},
providing engineers a fast and safe prototyping tool.

\vspace{-0.5mm}
\section{Prototypes of resilient processing cores\label{sec:experience}}
\vspace{-0.5mm}

% Formalising the design specifications is a good approach to the development of a
% resilient system. Specifications, represented via DSL, can indeed be verified and
% simulated, targeting an error-free system. Workcraft~\cite{workcraft_web} provides the support for different models, this can help the design for resiliency. Conditional partial order graphs (CPOG)~\cite{ISA-formal} (available in the Workcraft framework), for instance, can support the development of processor instruction sets.
% In Figure \ref{fig:screenshot}, all the steps from the
% specification of each instruction, to the synthesis of the final microcontroller are
% depicted.

% Instructions can be specified in the form of algebraic equations at first, then the
% graphs can be imported, encoded and synthesised in the form of a microcontroller. The latter
% can be mapped with the usage of a technology library and eventually simulated: internal and
% external signals can be visualised via digital waveforms, where dependencies among signals
% are shown. we are developing a DSL for bridging the gap between formal methods toolsets such
% Rodin and hardware synthesis toolsets such as Workcraft. Having a scalable, yet
% reusable language would be beneficial for resiliency. The instruction correctness, indeed,
% could be simulated and formally verified, the partial order should be derived from the
% specifications.

We validated the design flow in Fig.~\ref{fig:screenshot} by fabricating two
ASIC prototypes: an asynchronous Intel 8051 core~\cite{rec-proc} and a dataflow
accelerator. Both were designed to survive in a wide range of voltages and supported
runtime reconfigurability.

Intel 8051 chip was fabricated in the 130nm process using the standard cell
library by STMicroelectronics, with the nominal supply voltage of 1.2V. Thanks
to runtime reconfigurability the chip could operate
in the voltage range 0.22-1.5V. Parts of the chip failed at certain voltages:
the SRAM failed below 0.89V, and the
conventional program counter was unreliable below 0.74V. However, the
processing core, designed using the presented methodology, operated
correctly down to 0.22V.

\begin{figure}[h!]
\begin{center}
  \includegraphics[width=0.84\linewidth]{FIG/ope-chip.pdf}
  \vspace{-3mm}
  \caption{Resiliency of asynchronous control under unstable voltage.}
  \label{fig:voltage-resiliency}
\end{center}
\vspace{-7mm}
\end{figure}

The dataflow accelerator prototype (our DATE'17 University Booth demo) was
designed as an asynchronous dataflow processor with reconfigurable pipeline depth.
It was fabricated using the TSMC~90nm CMOS technology for low-power applications
and retained functionality in the range 0.34-1.6V.
Fig.~\ref{fig:voltage-resiliency} shows the resiliency of the chip, which adapts
to varying voltage by slowing down and reducing power consumption.

\begin{thebibliography}{12}
\vspace{-1mm}

\bibitem{calinescu2010formal}
R. Calinescu, S. Kikuchi. \emph{``Formal methods @ runtime''}. In Proceedings of the Monterey Workshop, Pages 122–135, Springer, 2010.

\bibitem{DBLP:journals/corr/BhattacharyyaMP16}
A. Bhattacharyya, A. Mokhov, K. Pierce. \emph{``An Empirical Comparison of Formalisms for Modelling and Analysis of Dynamic Reconfiguration of Dependable Systems''}. Formal Aspects of Computing, Springer, 2016.

\bibitem{ISA-formal}
A. Mokhov et al.
\emph{``Synthesis of processor instruction sets from high-level ISA specifications''}. IEEE Transaction on Computers 2014, vol. 63(6).

\bibitem{workcraft_web}
    \textsc{Workcraft} framework homepage: \url{http://www.workcraft.org/}.

\bibitem{rec-proc}
  A. Mokhov, M. Rykunov, D. Sokolov, A. Yakovlev.
  \emph{``Design of Processors with Reconfigurable Microarchitecture''}.
  J. Low Power Electronics Application, 2014, vol. 4(1), pp. 26-43.

\bibitem{HudakDSLs}
  P. Hudak.
  \emph{``Modular Domain Specific Languages and Tools''}.
  Proceedings of the International Conference on Software Reuse, 1998, p. 134.

\bibitem{STG}
J. Cortadella et al. \emph{``Logic synthesis for asynchronous controllers and interfaces''}, Springer, 2012.

\bibitem{DFS}
  D. Sokolov, I. Poliakov, A. Yakovlev. \emph{``Analysis of static data flow structures''}. Fundamenta Informaticae, vol. 88(4), pp. 581-610, 2008.

\bibitem{EventB}
  J-R. Abrial. \emph{Modeling in Event-B: system and software engineering}. Cambridge University Press, 2010.

\bibitem{WadlerMonads}
  P. Wadler.
  \emph{``Monads for functional programming''}.
  Advanced Functional Programming, Bastad Spring School, 1995.

% \bibitem{OPE}
%   C. Guo, W. Luk, S. Weston:
%   \emph{``Pipelined reconfigurable accelerator for ordinal pattern encoding''}.
%   IEEE Conference on Application-specific Systems, Architectures and Processors~(ASAP), 2014,
%   pp.~194--201.

% \bibitem{Weyns:2012:SFM:2347583.2347592}
% D. Weyns, M. U. Iftikhar, D. G. de la Iglesia, T. Ahmad. \emph{``A
% survey of formal methods in self-adaptive systems''}. In Proceedings of the International C* Conference on
% Computer Science and Software Engineering, C3S2E ’12, (New York, NY, USA), Pages 67–79, ACM, 2012.

% \bibitem{kikuchi2010configuration}
% S. Kikuchi, S. Tsuchiya. \emph{``Configuration procedure synthesis for complex systems using model finder''}.
% In Proceedings of the 15th IEEE International Conference on Engineering of Complex Computer
% Systems (ICECCS), Pages 95–104, IEEE, 2010.

% \bibitem{singh1999formal}
% S. Singh, C. J. Lillieroth. \emph{``Formal verification of reconfigurable core''}.
% Seventh Annual IEEE Symposium on Field-Programmable Custom Computing Machines, 1999. FCCM’99.
% Proceedings. , Pages 25–32, IEEE, 1999.

% \bibitem{Babin2016}
% G. Babin, Y. A{\"i}t-Ameur, N. K. Singh, M. Pantel. \emph{``A System Substi-
% tution Mechanism for Hybrid Systems in Event-B''}.  In Proceedings of the International
% Conference on Formal Engineering Methods, Pages 106-121. Springer International Publishing, 2016.

\end{thebibliography}

\end{document}
